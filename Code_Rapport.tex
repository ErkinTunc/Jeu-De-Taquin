\documentclass{article}
\usepackage[french]{babel}
\usepackage[utf8]{inputenc}
\usepackage[T1]{fontenc}
\usepackage{amssymb}
\usepackage{mathtools}
\usepackage{listings}
\usepackage{hyperref}

\title{L2 INFORMATIQUE – PROGRAMMATION AVANCEE \\ PROJET : JEU DE TAQUIN}
\author{Erkin Tunç Boya}
\date{Décembre 2023-2024}

\begin{document}

\maketitle
\tableofcontents
\newpage


\section{Jeu De Taquin}
Le jeu de taquin consiste en un
plateau composé de 4 lignes et 4 colonnes. Ce plateau comporte 15 pavés, une case est donc
laissée libre. Chaque pavé est numéroté avec un nombre unique compris entre 1 et 15.
Les pavés sont disposés sur le plateau d’une maniére désordonnée. Le but est de se ramener a la
situation ou les nombres sont triés dans un ordre croissant. Seuls les pavés adjacents a la case libre
peuvent etre placés sur cette case. Les déplacements autorisés sont les déplacements d’une case de
maniére horizontale ou verticale.

%Peut-etre on peut ajouter des images pour montrer le logique de jeu (comme le fichier de projet)

\section{Localisation}
Dans ce projet, il y a 4 fichiers principaux : 3 qui contiennent du codes de jeu et un fichier Makefile.
\\
\begin{itemize}
	\item Code\_Taquin.c
	\item Code\_Taquin.h
	\item main.c  \\
	\item Makefile \\
\end{itemize}


Le fichier Taquin.c contient le code des fonctions.
\\

Le fichier Taquin.h contient le structure de jeu et les nom de fonction(pour les appeller).\\

Le fichier main.c contient le code de menu(interface graphique). Dans celui-ci, on utilise le code des deux autres fichiers.


\section{Structure De Données}

\subsection{GameBoard}
La totalité du jeu est contruit à partir de la structure suivant.
\\
Dans cette structure il y a : \\

\begin{itemize}
	\item size (une taille de tableau)
	\item board (un pointeur qui pointe sur un tableau à 2 dimension.)
	\item emptyLin (l'indice de ligne de la case vide)
	\item empyCol (l'indice de colonne de la case vide)
\end{itemize}

\newpage

\section{Les Fonctions De Jeu}

\subsection{CreationTab}
Cette fonction crée le premier plateau de jeu qui est comme ci-dessous.(Nous supposons que size est égale à 4) \\

\begin{center}
	T =
	\begin{tabular}{|c|c|c|c|} \hline
	 1 & 2 & 3 & 4 \\  \hline
	 5 & 6 & 7 & 8 \\  \hline
	 9 & 10 & 11 & 12 \\ \hline
	13 & 14 & 15 & 0 \\ \hline
	\end{tabular}
\end{center}
 

\subsection{AffectFichier}
Cette fonction nous permet de continuer un jeu qui est deja sauvgardé au sein d'un fichier, grace à une lecture. Avant d'utiliser cette fonction, la taille et le plateau (avec la bonne taille) doit être définie grace à la structure mentionnée précédemment.

\subsection{EcrireFichier}
Cette fonction nous permet d'écrire le contenu du plateau (jeu) dans un fichier .txt, afin de le saugarder.
Cette fonction n'est pas disponible dans le code.

\subsection{mouve}
Cette fonction permet le déplacement de la case vide. Nous avons décidé d'utiliser des pointeurs des \textit{pointeurs de fonction} pour cette fonction. Pour cela nous avons crée 4 fonctions.	

\begin{enumerate}
	\item X\_gauche 			
	\item X\_droite 			
	\item X\_VersLeHaut		
	\item X\_VersLeBas		
\end{enumerate}

Ces quatres fonctions nous permettent de changer l'indice colonne et ligne de la case vide(qui est represante par 0). 
	

\subsection{Melange}
Cette fonction nous permet de melanger un plateau. Il utilise le pointeur de structure comme premier argument et il modifie le placement de la case vide aussi.

\newpage

\section{Fonction Pour Affichage et Pour Tester}

\subsection{AfficheTab2k}
Cette fonction permet d'afficher un tableau à 2 dimensions. De plus, il mets un 'X' à la place du numéro 0 dans la case vide.

\subsection{clearScreen}
Cette fonction efface les fragements de texte d'afficher sur le terminal qui ne sont pas en lien avec le jeu du Taquin. Il fonctionne sur Linux et Windows.

\section{Interface Graphique}
Pour l'interface, le jeu utilise le terminal. Il affiche le texte souhaité sur le terminal et efface l'autre texte avec la fonction clearScreen(). De plus, le jeu utilise toutes les fonctions mentionnés précedemment tout au long du jeu.
\\
Le menu du jeu contient 4 partie :

\subsection{Partie I : Jouer}
En partie I le joueur joue le jeu grace à des boucles et les fonctions suivantes:  

\begin{itemize}
	\item clearScreen()
	\item AfficheTab2k()
	\item mouve()
\end{itemize}


\subsection{Partie II : jouer a une partie sauvegardée}
En partie II le joueur peut choisir la difficulté.
Il y a 3 choix possible: facile, moyenne, difficile.
Chaque choix de difficulté représente un fichier prédéfinie dans le dossier.
Grace à la fonction AffectFichier(), Le jeu peut affecter les valeurs des fichier sur notre structure.
Par ailleurs, la fonction clearScrean() est utilisé.
\\ \\
Les fichiers correspondant aux choix sont les suivants: 

\begin{itemize}
	\item facile $\leftrightarrow$ facile.txt
	\item moyenne $\leftrightarrow$ moyenne.txt
	\item difficile $\leftrightarrow$ difficile.txt
\end{itemize}


\subsection{Partie III : Les options}
En partie III le joueur peut choisir le taille du plateau. L'interface lui permet de choissir le taille 4 , 5 et 6.

\subsection{Partie IV : Quitter}
Dans cette partie, le joueur a la possibilté de quitter le jeu.

\section{Makefile}
Le code du fichier Makefile comme suivant nous permet de créer l'executable(Jeu\_de\_Taquin) de notre projet.
TARGET represente le nom d'executable.
\begin{lstlisting}
CC=gcc
FLAGS= -Wall
OBJECTS= Code_Taquin.o main.o
TARGET=Jeu_de_Taquin

all: $(TARGET)

$(TARGET): $(OBJECTS)
	$(CC) $(FLAGS) -o $@ $(OBJECTS)

Code_Taquin.o: Code_Taquin.c Code_taquin.h
	$(CC) $(FLAGS) -c $<

main.o: main.c Code_taquin.h
	$(CC) $(FLAGS) -c $<
\end{lstlisting}

\section{Notes du Developeur}
Le Depot Git : \url{https://github.com/ErkinTunc/Jeu-De-Taquin} \\

\end{document}
