\documentclass{article}
\usepackage[french]{babel}
\usepackage[utf8]{inputenc}
\usepackage[T1]{fontenc}
\usepackage{amssymb}
\usepackage{mathtools}

\title{L2 INFORMATIQUE – PROGRAMMATION AVANCEE \\ PROJET : JEU DE TAQUIN}
\author{Erkin Tunç Boya}
\date{Décembre 2023-2024}

\begin{document}

\maketitle
\tableofcontents
\newpage


\section{Jeu De Taquin}
Le jeu de taquin consiste en un
plateau composé de 4 lignes et 4 colonnes. Ce plateau comporte 15 pavés, une case est donc
laissée libre. Chaque pavé est numéroté avec un nombre unique compris entre 1 et 15.
Les pavés sont disposés sur le plateau d’un maniére désordonnée. Le but est de se ramener a la
situation ou les nombres sont triés de maniére croissante. Seuls les pavés adjacents a la case libre
peuvent etre placés sur cette case. Les déplacements autorisés sont le déplacement d’une case de
maniére horizontale ou verticale.

%Peut-etre on peut ajouter des images pour montrer le logique de jeu (comme le fichier de projet)

\section{Structure De Données}

\subsection{Case}

\subsection{GameBoard}


\section{Les Fonctions De Jeu}

\subsection{CreationTab}


\section{Fonction Pour Tester}

\subsection{AfficheTab2k}

\end{document}
