\documentclass{article}
\usepackage[french]{babel}
\usepackage[utf8]{inputenc}
\usepackage[T1]{fontenc}
\usepackage{amssymb}
\usepackage{mathtools}

\title{L2 INFORMATIQUE – PROGRAMMATION AVANCEE \\ PROJET : JEU DE TAQUIN}
\author{Erkin Tunç Boya}
\date{Décembre 2023-2024}

\begin{document}

\maketitle
\tableofcontents
\newpage


\section{Jeu De Taquin}
Le jeu de taquin consiste en un
plateau composé de 4 lignes et 4 colonnes. Ce plateau comporte 15 pavés, une case est donc
laissée libre. Chaque pavé est numéroté avec un nombre unique compris entre 1 et 15.
Les pavés sont disposés sur le plateau d’un maniére désordonnée. Le but est de se ramener a la
situation ou les nombres sont triés de maniére croissante. Seuls les pavés adjacents a la case libre
peuvent etre placés sur cette case. Les déplacements autorisés sont le déplacement d’une case de
maniére horizontale ou verticale.

%Peut-etre on peut ajouter des images pour montrer le logique de jeu (comme le fichier de projet)

\section{Localisation}
Dans ce projet il y a 3 fichier qui contiennent de codes de jeu et une Makefile.
\\
\begin{itemize}
	\item Code\_Taquin.c
	\item Code\_Taquin.h
	\item main.c  \\
	\item Makefile \\
\end{itemize}


Le fichier Taquin.c contient le code des fonctions.
\\

Le fichier Taquin.h contient le structure de jeu et les nom de fonction(pour les appeller).\\

Le fichier main.c contient le code de menu(interface graphique). Dans cette fichier on utilise les codes d'autre fichier.


\section{Structure De Données}

\subsection{GameBoard}
Tous le jeu est contruit sur cette structure.
\\
Dans cette structure il y a : \\

\begin{itemize}
	\item size (taille de tableau)
	\item board (le pointeur qui pointe le tableau 2 dim.)
	\item emptyLin (le location de colonne de case vide)
	\item empyCol (le location de colonne de case vide)
\end{itemize}


\section{Les Fonctions De Jeu}

\subsection{CreationTab}
Cette fonction crée le premier table de jeu qui est comme ci-dessous.

%ici ajoute une matrice  4x4 qui a le nombre croissante 

\subsection{AffectFichier}
Cette fonction nous permet de continuer un jeu qui est deja sauvgardé sur le fichier. Grace a lisage d'un fichier.

\subsection{EcrireFichier}
Cette fonction nous permet de ecrire le contenu de tableau(jeu) a la fichier donc il nous permet de saugarder.
Mais cette fonction n'est pas dans le code car ce n'est pas preferé utilisation par nos developpeurs.

\subsection{mouve}
Dans cette fonction nous avons utiliser \textit{pointeurs de fonction}. Pour cela on a crée 4 fonction.	

\begin{enumerate}
	\item X\_gauche 			
	\item X\_droite 			
	\item X\_VersLeHaut		
	\item X\_VersLeBas		
\end{enumerate}

Cette quatre fonctions nous permétont de changer le location de case vide(qui est represante par '0'). 
	

\subsection{Melange}
Cette fonction nous permet de melange un tableau par hazard.

\section{Fonction Pour Affichage et Pour Tester}

\subsection{AfficheTab2k}
Cette fonction affiche le tableau 2 dimension avec un bon representation. Enplus, il affiche 'X' a place de numero 0 qui represente le case vide dans le jeu.

\subsection{clearScreen}
Cette fonction efface les ecritures sur le terminal pour Linux et Windows systeme d'operations

\section{Interface Graphique}
Pour l'interface le jeu utilise le terminal. Il affiche ces ecritures sur le terminal et efface avec le fonction clearScreen(). Enplus le jeu utiliser touses des fonctions (sauf EcrireFichier).
\\
Le menu de jeu contient 3 partie :

\subsection{Partie I : Jouer}
En partie I le joueur joue le jeu grace aux boucles et des fonction qui sont 

\begin{itemize}
	\item clearScreen()
	\item AfficheTab2k()
	\item mouve()
\end{itemize}


\subsection{Partie II : jouer a une partie sauvegardee}
En partie II le joueur peut choissir le difficulte.
Il y a 3 choix "facille,moyenne,difficile".
Chaque choix represante une fichier dans le docier.
Grace a focntion AffectFichier(). Leu jeu peut affecter les valeurs de fichier sur notre structure.
Enplus le fonction clearScrean() est utilisé.
\\ \\
Les équivalents fichier des choix sont ci-dessous : 

\begin{itemize}
	\item facile $\leftrightarrow$ facile.txt
	\item moyenne $\leftrightarrow$ moyenne.txt
	\item difficile $\leftrightarrow$ difficile.txt
\end{itemize}


\subsection{Partie III : Les options}
En partie III le joueur peut choisir le taille de tableau. L'interface lui permet de choissir le taille 4 , 5 et 6.

\subsection{Partie IV : Quitter}
Dans cette partie le joueur peut quitter le jeu grace a un button choissi

\section{Makefile}

\end{document}
