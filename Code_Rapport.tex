\documentclass{article}
\usepackage[french]{babel}
\usepackage[utf8]{inputenc}
\usepackage[T1]{fontenc}
\usepackage{amssymb}
\usepackage{mathtools}

\title{L2 INFORMATIQUE – PROGRAMMATION AVANCEE \\ PROJET : JEU DE TAQUIN}
\author{Erkin Tunç Boya}
\date{Décembre 2023-2024}

\begin{document}

\maketitle
\tableofcontents
\newpage


\section{Jeu De Taquin}
Le jeu de taquin consiste en un
plateau composé de 4 lignes et 4 colonnes. Ce plateau comporte 15 pavés, une case est donc
laissée libre. Chaque pavé est numéroté avec un nombre unique compris entre 1 et 15.
Les pavés sont disposés sur le plateau d’un maniére désordonnée. Le but est de se ramener a la
situation ou les nombres sont triés de maniére croissante. Seuls les pavés adjacents a la case libre
peuvent etre placés sur cette case. Les déplacements autorisés sont le déplacement d’une case de
maniére horizontale ou verticale.

%Peut-etre on peut ajouter des images pour montrer le logique de jeu (comme le fichier de projet)

\section{Localisation}
Dans ce projet il y a 3 fichier qui contiennent de codes
//
\begin{itamize}


\section{Structure De Données}

\subsection{GameBoard}


\section{Les Fonctions De Jeu}

\subsection{CreationTab}
Cette fonction crée le premier table de jeu qui est comme ci-dessous.

%ici ajoute une matrice  4x4 qui a le nombre croissante 

\subsection{AffectFichier}
Cette fonction nous permet de continuer un jeu nous avons sauvgardé. Grace a lisage d'un fichier

\subsection{EcrireFichier}
Cette fonction nous permet de ecrire le contenu de tableau(jeu) a la fichier donc il nous permet de saugarder.
Mais cette fonction n'est pas dans le code car ce n'est pas preferé utilisation par nos developpeurs.

\subsection{mouve}
Dans cette fonction nous avons utiliser \textit{pointeurs de fonction}. Pour cela on a crée 4 fonction.	

\begin{enumerate}
	\item X\_gauche 			
	\item X\_droite 			
	\item X\_VersLeHaut		
	\item X\_VersLeBas		
\end{enumerate}

Cette quatre fonctions nous permétont de changer le location de case vide(qui est represante par '0'). 
	

\subsection{Melange}
Cette fonction nous permet de melange un tableau par hazard.

\section{Fonction Pour Tester}

\subsection{AfficheTab2k}
Cette fonction affiche le tableau 2 dimension avec un bon representation. Enplus, il affiche 'X' a place de numero 0 qui represente le case vide dans le jeu.

\subsection{clearScreen}
Cette fonction efface les ecritures sur le terminal pour Linux et Windows systeme d'operations

\section{Interface Graphique}
Pour l'interface le jeu utilise le terminal. Il affiche ces ecritures sur le terminal et efface avec le fonction clearScreen().
//
Le menu de jeu contient 3 partie...

\section{Makefile}

\end{document}
